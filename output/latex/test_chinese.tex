\documentclass[UTF8, a4paper, 12pt]{ctexart}

% 基础包
\usepackage{amsmath}
\usepackage{amssymb}
\usepackage{enumitem}
\usepackage{booktabs}
\usepackage{tabularx}
\usepackage{graphicx}
\usepackage{geometry}
\usepackage{float}
\usepackage{hyperref}

% 代码高亮
\usepackage{listings}
\usepackage{xcolor}

% 页面设置
\geometry{
    left=2.5cm,
    right=2.5cm,
    top=2.5cm,
    bottom=2.5cm
}

% 代码样式设置
\lstset{
    basicstyle=\ttfamily\small,
    breaklines=true,
    showstringspaces=false,
    commentstyle=\color{gray},
    keywordstyle=\color{blue},
    stringstyle=\color{red},
    frame=single,
    numbers=left,
    numberstyle=\tiny\color{gray}
}

% 超链接设置
\hypersetup{
    colorlinks=true,
    linkcolor=blue,
    filecolor=magenta,      
    urlcolor=cyan,
    citecolor=green
}

\begin{document}


\section{中文测试文档}


\subsection{简介}

这是一个用于测试 MD2LaTeX 中文支持的示例文档。


\subsection{基本格式测试}


\subsubsection{文本格式}

这是普通文本,包含\textbf{粗体文本}和\emph{斜体文本}。


\subsubsection{列表测试}


\paragraph{有序列表}

\paragraph{无序列表}

\subsubsection{代码块测试}


\begin{lstlisting}[
    language=python,
    basicstyle=\ttfamily\small,
    breaklines=true,
    showstringspaces=false,
    commentstyle=\color{gray},
    keywordstyle=\color{blue},
    stringstyle=\color{red},
    frame=single,
    numbers=left,
    numberstyle=\tiny\color{gray}
]
def hello_world():
    print("你好,世界!")
    return "Hello, World!"

\end{lstlisting}


\subsubsection{数学公式测试}


行内公式:$\alpha + \beta = \gamma$


块级公式:

\begin{equation}
\int_{-\infty}^{\infty} e^{-x^2} dx = \sqrt{\pi}
\end{equation}


\subsubsection{表格测试}


\begin{table}[H]
    \centering
    \caption{1}
    \label{tab:table1}
    \begin{tabular}{|l|l|l|}
        \toprule
        \textbf{姓名} & \textbf{年龄} & \textbf{职业} \\
        \midrule
        张三 & 25 & 工程师 \\
        李四 & 30 & 设计师 \\
        王五 & 28 & 产品经理 \\
        \bottomrule
    \end{tabular}
\end{table}

\subsubsection{引用测试}


\begin{quote}
\kaishu 
这是一个中文引用块。


包含多行内容的引用。

\end{quote}


\subsection{结论}


MD2LaTeX 工具能够很好地处理中文内容,包括各种格式和特殊字符。


\end{document}
